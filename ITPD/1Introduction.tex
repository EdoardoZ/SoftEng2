\section{Introduction}

\subsection{Revision History}
\begin{tabular}{| l | l | p{10cm} |}
\hline
\textbf{Version} & \textbf{Date} & \textbf{Changes}\\
\hline
1.0-RC1 & 15/01/2017 & First deadline release.\\
\hline
\end{tabular}

\subsection{Purpose and Scope}
This document contains indications about how to proceed during the testing phase and the step by step integration of the PowerEnJoy system components. The expected audience of this document are test developers as well as project managers, who need to coordinate testing with developing activities. 

\subsection{List of Definitions and Abbreviations}
\subsubsection{Acronyms, Abbreviations}
\begin{itemize}
    \item GPS = Global Positioning System
    \item ORAA = Onboard Ride Assistant App
    \item PEJ = PowerEnJoy
   %\item PPP = Payment Processor Provider
    \item SBL = Service Business Logic
    \item SPA = Special Parking Area
    \item UDMA = User Device Mobile App
\end{itemize}

\subsubsection{Definitions}
\begin{itemize}
    \item \textbf{Car:} every vehicle, which respects the requirements, that the system allows the users to use.
    \item \textbf{Onboard Ride Assistant App:} software logic that manage what is shown on the Car screen.
   %\item\textbf{PowerUser:} user who is already registered to the service and is currently logged in.
    \item \textbf{Safe Parking Area:} pre-defined areas (i.e. streets) where a user is allowed to park.
    \item \textbf{Service Business Logic:} software logic of the service the user do not directly interact with.
    \item \textbf{Special Parking Area:} pre-defined areas (i.e. streets) where a user is allowed to park and where the batteries of the Cars can be plugged into the power grid.
    %\item \textbf{Total Base Fare:} amount of money that the user pays for the ride duration only. It does not include any discount or penalty for any other specific condition.
    %\item \textbf{Total Ride Fare:} amount of money that the user pays. It includes any discount or penalty for any other specific condition satisfied during the ride.
    \item \textbf{User Device Mobile App:} mobile application the user will interact with on his mobile phone.
\end{itemize}


\subsection{List of Reference Documents}
\begin{itemize}
    \item Project’s Assignment document: AA 2016-2017 Software Engineering 2 - Project goal, schedule, and rules.
    \item Software Engineering 2: "PowerEnJoy" Requirements Analysis and Specification Document.
    \item Software Engineering 2: "PowerEnJoy" Design Document.
    \item Xamarin test cloud: \url{https://www.xamarin.com/test-cloud}
    \item xUnit: \url{https://en.wikipedia.org/wiki/XUnit}
    \item HockeyApp: \url{https://hockeyapp.net/#s}
\end{itemize}