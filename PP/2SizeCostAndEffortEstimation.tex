\section{Size, Cost and Effort Estimation}

\subsection{Size Estimation: Function Points}

\noindent
{
\begin{center}
\begin{tabular}{ lccc }
\hline
\multicolumn{4}{c}{\textbf{UFP Complexity Weights}} \\
\hline
\textbf{Function type} & \textbf{Low} & \textbf{Average} & \textbf{High} \\ 
\hline
Internal Logic Files & 7 & 10 & 15 \\
\hline
External Interface Files & 5 & 7 & 10 \\
\hline
External Inputs & 3 & 4 & 6 \\ 
\hline
External Output & 4 & 5 & 7 \\
\hline
External Inquiries & 3 & 4 & 6 \\
\hline
\end{tabular}
\end{center}
}



\subsubsection{Internal Logical File}
\begin{itemize}
    \item \textbf{Safe Areas Data}\\
    It consist of a set of Coordinates (\textit{Longitude}, \textit{Latitude}), so we consider it with a low weight.
    \hfill
    \textbf{7 FPs}
    
    \item \textbf{SPA data}\\
    As for the Safe Area this is a list of Coordinates, it can be considered with a low weight.
    \hfill
    \textbf{7 FPs}
    
    \item \textbf{User Data}\\
    It has different fields (Name, Surname, E-mail, Password, Credit Card, Driving Licence) but some of those require encryption and special check like Password, Credit Card and Driving Licence, because of that we consider it with a average weight.
    \hfill
    \textbf{10 FPs}
    
    \item \textbf{Car Data}\\
    As for the user, it is one entity and contains car attributes, so we consider it with a low weight.
    \hfill
    \textbf{7 FPs}
    
    \item \textbf{Discount Data}\\
    It's an entity where all discounts are stored and so is pretty simple, we consider it with a low weight.
    \hfill
    \textbf{7 FPs}
    
    \item \textbf{Booking Data}\\
    It has a simple structure: contains (bookingTime, CarID, UserID), so it will be considered with a low weight.
    \hfill
    \textbf{7 FPs}
    
    \item \textbf{Ride Data}\\
    It has a simple structure: contains (rideStart, rideEnd, CarID, UserID), so it will be considered with a low weight.
    \hfill
    \textbf{7 FPs}
\end{itemize}


\subsubsection{External Interface File}
\begin{itemize}
    \item \textbf{Localization Service}\\
    Contains the Coordinates of the vehicle, since it has a simple structure we consider it with a low weight.
    \hfill
    \textbf{5 FPs}
    
    \item \textbf{Map Retrieval}\\
    Contains a complex structure acquired from the third party map provider, so we assigned an high weight.
    \hfill
    \textbf{10 FPs}
    
    \item \textbf{Payment Data}\\
    Contains the Payment Data, since this information needs to be safely stored we consider it with an average weight.
    \hfill
    \textbf{7 FPs}
\end{itemize}


\subsubsection{External Inputs}
\begin{itemize}
    \item \textbf{Account Creation}\\
    It involves only the Account Manager but some data needs to be carefully checked, we assign an average weight.
    \hfill
    \textbf{4 FPs}
    
    \item \textbf{Login/Logout}\\
    They are simple operation: involves only the Account Manager.
    \hfill
    2x3=\textbf{6 FPs}
    
    \item \textbf{Profile Management}\\
    As for the Account Creation we may need to do some check, so we assign an average weight.
    \hfill
    \textbf{4 FPs}
    
    \item \textbf{Car Booking}\\
    It involves several entities, because of that we assign an high weight.
    \hfill
    \textbf{6 FPs}
    
    \item \textbf{Reservation Deletion}\\
    It's a quite simple operation: set the booked car as free and delete the booking instance, so we consider it with a low weight.
    \hfill
    \textbf{3 FPs}
    
    \item \textbf{Car Unlocking}\\
    It involves several entities and some conditions must be checked, we assign an high weight.
    \hfill
    \textbf{6 FPs}
\end{itemize}


\subsubsection{External Outputs}
\begin{itemize}
    \item \textbf{Ride and Base Fare}\\
    The Base Fare value must be shown in real time, Ride Fare is calculated at the end of the ride from the Base Fare value, because of that we assign to them a low weight.
    \hfill
    2x4=\textbf{8 FPs}
    
    \item \textbf{Successful Book Notification}\\
    It's a simple notification sent to the user, we consider it with a low weight.
    \hfill
    \textbf{4 FPs}
    
    \item \textbf{Penalty Notification}\\
    This notify the user a bill him for the penalty amount, since it involves also the payment service we assign an average weight.
    \hfill
    \textbf{5 FPs}
    
\end{itemize}


\subsubsection{External Inquiries}
\begin{itemize}
    \item \textbf{Retrieve Booked Car}\\
    It's a quite simple query, we only need to look for the right Car ID, so we consider it with a low weight.
    \hfill
    \textbf{3 FPs}
    
    \item \textbf{SPA List}\\
    Retrieve a list of closest SPAs, it's quite simple we assign a low weight.
    \hfill
    \textbf{3 FPs}
    
    \item \textbf{Closest SPA}\\
    Retrieve the closest SPA, to a given position, that satisfy our criteria, this is a simple function and for that we assign a low weight.
    \hfill
    \textbf{3 FPs}
    
    \item \textbf{Discount Available}\\
    It retrieve the discount available and check if they are applicable, so we assign an average weight.
    \hfill
    \textbf{4 FPs}
    
    \item \textbf{Map with Cars Position}\\
    This needs to retrieve both map and Cars position, because of that we consider  it with an average weight.
    \hfill
    \textbf{4 FPs}
\end{itemize}


\subsubsection{Summary}

\begin{center}
\begin{tabular}{ |l|c| }
\hline
\textbf{Function Type} & \textbf{Total FPs} \\
\hline
Internal Logic Files & 7+7+10+7+7+7+7=\textbf{52} \\
\hline
External Interface Files & 5+10+7=\textbf{22}\\
\hline
External Inputs & 4+6+4+6+3+6=\textbf{29} \\ 
\hline
External Output & 8+4+5=\textbf{17} \\
\hline
External Inquiries & 3+3+3+4+4=\textbf{17} \\
\hline
Total UFPs & \textbf{137}\\
\hline
\end{tabular}
\end{center}



\subsection{Cost and Effort Estimation: COCOMO II}
\subsubsection{Scale Drivers}
\begin{itemize}
    \item \textbf{Precedentedness}\\
    The product belongs to a category which is not newly to our company. We expect to reuse most techniques and methodologies from previous experiences or well known patterns. However some features regarding the automotive IoT will certainly result in some challenges to be overcome.
    \hfill Nominal(3.72)
    
    \item \textbf{Development Flexibility}\\
    The customer imposed some requirements over high-level end-user experience. So we have enough freedom in designing inner components but the expected distributed characterization of the solution will certainly impose lots of constraints regarding internal subsystems interactions as well as external components interfacing.
    \hfill Low(4.05)
    
    \item \textbf{Architecture / Risk Resolution}\\
    To prevent major compatibility problems across various parts of the solution we intend to give strict guidelines regarding its architecture perspective. Such a choice will give relevant robustness to the solution design against unexpected issues affecting development phase.
    \hfill High(2.83)
    
    \item \textbf{Team Cohesion}\\
    Most of members of the team have previous experience working together at the same product. We wont expect the addition of new members will disrupt inner interpersonal relationships. Moreover each member has reasonably no reasons to leave the team.
    \hfill Very High(1.10)
    
    \item \textbf{Process Maturity}\\
    As a software production company working with cutting-edge technologies we're continuously in search of new approaches to address new design scenarios, therefore major production phases are subject to processes for which no maturity rating is available.
    \hfill Low(6.24)
\end{itemize}

\subsubsection{Cost Drivers}
\begin{itemize}
    \item Product
    \begin{itemize}
        \item \textbf{Required Software Reliability}\\
        As stated in RASD, the customer has not expressed any reliability need so we expect to implement basics reliability oriented patterns in order to invest less effort as possible on this point even keeping reliability at a more than sufficient level.
        \hfill Low(0.88)
        
        \item \textbf{Database Size}\\
        The solution will not require more than a middle-low amount of data to be managed as long as the customer does not intend to keep temporary data for other reasons like mining or selling. In any case the amount of real-time managed data does not grow over some predictable levels.
        \hfill Nominal(1.00)
        
        \item \textbf{Product Complexity}\\
        The solution will manage a large amount of devices that need to be concurrently connected and synchronized by many different protocols. We expect to decompose the product into many different components to help managing that level of complexity.
        \hfill High(1.15)
        
        \item \textbf{Developed for Reusability}\\
        The abstraction levels of many solution elements will be high to overcome the front-end volatility. Moreover, the application scenario at this historical moment is quite common so we expect much of our work may come handy in future if we plan to reuse some elements.
        \hfill Very High(1.29)
        
        \item \textbf{Documentation Match to Life-Cycle Needs}\\
        We provide deeply detailed documentation for each step of solution life-cycle. The suitability of those documents should give a valuable speed-up in terms of effort.
        \hfill High(1.06)
    \end{itemize}
    
    \item Personnel
    \begin{itemize}
        \item \textbf{Analyst Capability}\\
        Analysts involved in this project have proven experience into providing very good solutions to various kind of problems this product could arise. 
        \hfill High(0.83)
        
        \item \textbf{Programmer Capability}\\
        We expect most of the team will be able to work without major problems on the coding of required features.
        \hfill Nominal(1.00)
        
        \item \textbf{Personnel Continuity}\\
        We have very good reasons to grant for sure that each member of the team will remain part of this project until the end of the academic course this project relates to.
        \hfill High(0.92)
        
        \item \textbf{Applications Experience}\\
        The whole team has an average experience in working onto this specific kind of applications. To give a more realistic and tolerant effort prediction we prefer to keep its rating at a low level.
        \hfill Low(1.10)
        
        \item \textbf{Platform Experience}\\
        Given that the deployment platform of the solution is still undefined we prefer to keep this rating at a low level to better adapt to new platforms eventually which would eventually serve better the architecture choices.
        \hfill Low(1.10)
        
        \item \textbf{Language and Toolset Experience}\\
        Most of the team members have a very good familiarity with any kind of software development framework and its related environments. We expect am high level of knowledge sharing across team members.
        \hfill High(0.88)
    \end{itemize}
    
    \item Platform
    \begin{itemize}
        \item \textbf{Time Constraint}\\
        No kind of requirements from the customer. However we prefer to keep the solution as usable as other similar competitors solutions.
        \hfill Nominal(1.00)
        
        \item \textbf{Storage Constraint}\\
        No kind of requirements from the customer. Since no further decisions will be taken on this point, we keep its effort rating to a standard level.
        \hfill Nominal(1.00)
        
        \item \textbf{Platform Volatility}\\
        We expect to base our solution on cutting-edge technologies which are subject to a well known rating of volatility. Also some subsystems will probably need to be reworked in future, especially in the set of user-premise elements subject to frequent upgrades.
        \hfill  Very High(1.30)
    \end{itemize}
    
    \item Project
    \begin{itemize}
        \item \textbf{Use of Software Tools}\\
        The use of powerful enterprise frameworks will give a valuable advantage in terms of the effort taken to configure and setup both development and production environments
        \hfill High(0.86)
        
        \item \textbf{Multi-site Development}\\
        The whole team is collocated in the same site. Any interaction with the customer will not take any effort in time since the customer location is the same of our company.
        \hfill Very High(0.84)
        
        \item \textbf{Required Development Schedule}\\
        Since we had no release timing requirement from the customer we keep a standard rating for this point.
        \hfill Nominal(1.00)
    \end{itemize}
\end{itemize}

\subsubsection{Expected effort summary}
Given the effort ($\epsilon$) equation, which is:
    \begin{equation*}
        \epsilon = A \times Size^E \times EAF
    \end{equation*}
    Where:
    \begin{itemize}
        \item A = 2.94 as in COCOMO II.2000

        \item Size: actual size of the product to be developed in terms of KSLOC (thousands of Source Lines Of Code).
        \begin{equation*}
            Size = \frac{UFP \times LF}{1000}
        \end{equation*}
        Where LF = 46 SLOCs as for the avarage value of J2EE applications.

        \item E : Exponent derived from the five Scale Drivers, with
        \begin{equation*}
            E = B + 0.01 \times \sum_{j=1}^5 SF_j
        \end{equation*}
        With B = 0.91 as in COCOMO II.2000

        \item EAF : Effort Adjustment Factor derived from the Cost Drivers, with
        \begin{equation*}
            EAF = \prod_{i=1}^N EM_i
        \end{equation*}
    \end{itemize}


Therefore the estimated effort is:
\begin{equation*}
        \epsilon = 2.94 \times (\frac{137\times 46}{1000})^{0.91+0.01\times 17.94} \times 1.15 = \textbf{29.4 person-month}
\end{equation*}

with an estimated duration of:
\begin{equation*}
        \delta = C \times \epsilon^{D+0.2\times(E-B)} = 
        \textbf{11.2 months}
\end{equation*}

\textit{Note: C = 3.67 and D = 0.28 as for COCOMO II.2000}\\\\

And a (minimum) required number of people of:
\begin{equation*}
        \pi = \lceil \epsilon / \delta \rceil = \textbf{4 persons}
\end{equation*}