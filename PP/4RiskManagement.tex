\section{Risk Management}
This section contains the contingency plans for the identified risks. To help understating the classification, we add a brief explanation of probability coefficients:
\begin{itemize}
    \item \textbf{Low Probability}: risk is unlikely to arise.
    \item \textbf{Moderate Probability}: risk should not arise, but it is possible it will.
    \item \textbf{High Probability}: risk is likely to arise.
\end{itemize}
and effect coefficients:
\begin{itemize}
    \item \textbf{Moderate Effect}: impact on project schedule should be limited.
    \item \textbf{Serious Effect}: large impact on the project that may end up in a small delay.
    \item \textbf{Critical Effect}: project delayed to face the new problems.
    \item \textbf{Catastrophic Effect}: the entire project may be cancelled.
\end{itemize}


\begin{tabular}{| l | p{7cm}  | l |  p{2.8cm} |}
\hline
\textbf{Risk} & \textbf{Risk Description} & \textbf{Probability} & \textbf{Effect}\\
\hline
R1 & Group member get ill and cannot work on the project for a short period of time. & Moderate & Moderate\\
\hline
R2 & Requirements are modified by stakeholders or contains errors. & Moderate & Critical\\
\hline
R3 & Milestones cannot be completed within the deadline. & Moderate & Critical\\
\hline
R4 & Architecture cannot scale well enough to handle an unexpected load. & Moderate & Critical\\
\hline
R5 & Failures in external components of the system. & Low & From Critical to Catastrophic\\
\hline
R6 & Damage on system platform hardware. & Low & Catastrophic\\
\hline
R7 & Bug on application. & Moderate & Moderate\\
\hline
R8 & New competitors enters the market. & Moderate & Serious\\
\hline
R9 & New laws impose unavoidable changing in the requirements. & Moderate & From Critical to Catastrophic\\
\hline
R10 & New stakeholders are discovered during the development of the system. & Moderate & From Moderate to Catastrophic\\
\hline
R11 & Agreement issues with one of the possible Internet provider that need to provide data connectivity to Cars. & Moderate & Critical\\
\hline
R12 & Positioning system (GPS, GLONASS, Galileo,...) availability is restricted and can no longer be used for users and Cars localization. & Low & Critical\\
\hline
\end{tabular}

\bigskip

\noindent
\begin{tabular}{| l | p{13cm}  |}
\hline
\textbf{Risk} & \textbf{Mitigation Strategy}\\
\hline
R1 & All members are aware of the main structure of the project, so in case of a missing member the others can take care of his work, and ensure work completion before deadlines; furthermore the schedule takes into account extra time to face unexpected events like this.\\
\hline
R2 & All the already completed milestone must be updated to reflect the changes; this takes a lot of effort if done during the implementation phase. Keeping extra time at each step of the project can be useful to limit the impact; a flexible system design can help speed up the updating phase; legally binding the customer to the original requirements can ensure he can only update the requirements to make them clearer, and never to add new ones.\\
\hline
R3 & Schedule should take into account extra time for each phase of the project in order to face unexpected events.\\
\hline
R4 & During the design phase the possibility of a peak in requests submitted to the service must be taken into account in order to design an architecture flexible enough to properly scale when needed.\\
\hline
R5 & If the failure is short a do not involve core functions the system may still be able to continue operating, and the impact of the failure would be seen only by a minor part of the user. If the failure last longer, most of the user might experience it, resulting in a loss of trust and the possibility of switching to a rival service. To prevent this the most fundamentals external components should be taken from a well-known and well-established provider, that can ensure its components will not face long outage.\\
\hline
R6 & To reduce the probability of damage on system hardware it it possible to deploy the system on a cloud infrastructure, whose hardware is located in different places. However privacy laws must be taken into account when selecting the places.\\
\hline
R7 & Any major bug should have been discovered during testing phase. Remaining bugs should not represent a big issue and can be address with a minor update on the application.\\
\hline
R8 & Depending on the economic and marketing strength of the competitor, it might be used to consider its functions and then decided whether invest to add new functions to the system we are developing, or to speed up the release in order to reach the market before the competitor, or organize a marketing campaign to ensure our system will have a better coverage than the one of the competitor.\\
\hline
R9 & Like \textit{R2} but with the impossibility of binding the customers. If the extent of the required modifications are too big it is possible that the system will no longer fit the budget even with the extra time described in R2. In this case a feasibility study should be done to evaluate whether increase the budget or, if possible, cut out some functions from the system.\\
\hline
R10 & Depending on the influence and the impact of interest of the new stakeholders we may end up in situation described in \textit{R2} or \textit{R9}.\\
\hline
\end{tabular}
\begin{tabular}{| l | p{13cm}  |}
\hline
R11 & Since the system relies on Internet data connectivity to send data back and forth from Cars to the central system, an issue with the Internet provider would restrain the system from being deployed. Due to number of Internet provider it is unlikely that an agreement cannot be reach with any of them; however exploring different solutions requires more time than just one solution. To ensure the lowest delay possible agreement with Internet provider should be done as soon as possible.\\
\hline
R12 & It is unlikely to happen due to the large impact that disabling the usage of a positioning system would have; however, especially after recent political events, it should be wise to rely on hardware that support at least a pair of different positioning systems.\\ 
\hline
\end{tabular}