\section{Introduction}
\subsection{Purpose}


\subsection{Scope}


\subsection{Definitions, Acronyms, Abbreviations}
\subsubsection{Acronyms, Abbreviations}
\begin{itemize}
    \item PEJ = PowerEnJoy
    \item SBL = Service Back-end Logic: software logic of the service the user do not directly interact with.
\end{itemize}
\subsubsection{Definitions}
\begin{itemize}
\item \textbf{Car:} every vehicle, which respects the requirements, that the system allows the users to use.
    \item \textbf{Special Parking Area:} pre-defined areas (i.e. streets) where a user is allowed to park and where the batteries of the Cars can be plugged into the power grid.
    \item \textbf{Total Base Fare:} amount of money that the user pays for the ride duration only. It does not include any discount or penalty for any other specific condition.
    \item \textbf{Total Ride Fare:} amount of money that the user pays. It includes any discount or penalty for any other specific condition satisfied during the ride.
\end{itemize}

\subsection{Reference Document}
\begin{itemize}
    \item Project?s Assignment document: AA 2016-2017 Software Engineering 2 - Project goal, schedule, and rules.
    \item IEEE Std 830-1998, IEEE Recommended Practice for Software Requirements Spec- ifications.
    \item Software Engineering 2: "PowerEnJoy" Requirements Analysis and Specification Document.
    \item UX diagram PDF from the course "Sistemi Informativi" of Politecnico of Milan.
\end{itemize}

\subsection{Document Structure}
This document is split in 6 parts:
\begin{itemize}
    \item \textbf{Architectural Design:} contains the different architectural views of the system.
    \item \textbf{Algorithm Design:} contains the pseudo-codes of the most relevant algorithmic parts of the system.
    \item \textbf{User Interface Design:} contains an overview of how the user interface on the system will look like.
    \item \textbf{Requirements Traceability:} shows the correlation among the requirements defined in the RASD and the design elements contained in this document.
    \item \textbf{Effort Spent:} contains the amount of time each member spent on the document, split between time spent working together and time spent on his own.
    \item \textbf{References:} contains all the references required by this document.
\end{itemize}