\section{Introduction}

\subsection{Purpose}
The goal of this document is to provide an overall description of PowerEnJoy software architectural design together with a brief overview of service peculiar algorithms. Several standard description diagrams will help the reader to better understand taken design decisions. This document is intended for an audience of software developers and project managers.

\subsection{Scope}
PowerEnJoy is an application for a new electric car-sharing service.\newline
User must be able to register to the service, with user-friendly mobile application, find a car nearby and book it. Once arrived near the car user must be able to unlock, enter and drive towards his destination.\newline
Inside the car, the user can interact with the car onboard ride assistant for getting details about his ride or driving directions.
\newline
Once the ride is ended the car will be locked by the system and user is billed.

\subsection{Definitions, Acronyms, Abbreviations}
\subsubsection{Acronyms, Abbreviations}
\begin{itemize}
    \item PEJ = PowerEnJoy
   % \item PPP = Payment Processor Provider
    \item SBL = Service Business Logic
    \item SPA = Special Parking Area
\end{itemize}
\subsubsection{Definitions}
\begin{itemize}
    \item \textbf{Car:} every vehicle, which respects the requirements, that the system allows the users to use.
    \item\textbf{PowerUser:} user who is already registered to the service and is currently logged in.
    \item \textbf{Safe Parking Area:} pre-defined areas (i.e. streets) where a user is allowed to park.
    \item \textbf{Service Business Logic:} software logic of the service the user do not directly interact with.
    \item \textbf{Special Parking Area:} pre-defined areas (i.e. streets) where a user is allowed to park and where the batteries of the Cars can be plugged into the power grid.
    \item \textbf{Total Base Fare:} amount of money that the user pays for the ride duration only. It does not include any discount or penalty for any other specific condition.
    \item \textbf{Total Ride Fare:} amount of money that the user pays. It includes any discount or penalty for any other specific condition satisfied during the ride.
\end{itemize}

\subsection{Internal Reference Documents}
\label{internal_ref}
\begin{itemize}
    \item Project’s Assignment document: AA 2016-2017 Software Engineering 2 - Project goal, schedule, and rules.
    \item Software Engineering 2: "PowerEnJoy" Requirements Analysis and Specification Document.
    \item See section \ref{external_ref} for external references.
\end{itemize}

\subsection{Document Structure}
This document is split in 6 parts:
\begin{itemize}
    \item \textbf{Architectural Design:} contains the different architectural views of the system.
    \item \textbf{Algorithm Design:} contains the pseudo-codes of the most relevant algorithmic parts of the system.
    \item \textbf{User Interface Design:} contains an overview of user interface and user experience.
    \item \textbf{Requirements Traceability:} shows the correlation among the requirements defined in the RASD and the design elements contained in this document.
    \item \textbf{Effort Spent:} contains the amount of time each member spent on the document, split between time spent working together and time spent on his own.
    \item \textbf{References:} contains all the external references required by this document.
\end{itemize}